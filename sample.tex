\documentclass{lab_report}

\title{A Sample Lab Report}
\author{Eric Zheng}
\sect{CSE}
\partners{Adrian Thamburaj}{Sam Uwe Alws}
\date{\today{}}

\begin{document}
\maketitle
\section{Introduction}
This is a sample document for the \verb|lab_report| \LaTeX class template provided by the \verb|lab_report.cls| file. This document goes over some of the basic usage and design from this class.

\section{Preamble}
The \verb|lab_report| class provides some custom information in the title. Generally, the preamble takes the form:
\begin{verbatim}
  \title{A Sample Lab Report}                % title
  \author{Eric Zheng}                        % author
  \sect{CSE}                                 % section: defaults to CSE
  \partners{Adrian Thamburaj}{Sam Uwe Alws}  % lab partners
  \date{\today{}}                            % date: defaults to today's date
\end{verbatim}

This information is also used to construct the page header. Note that the \verb|\partners| macro currently accepts exactly two arguments. If you need to add more partners, you can get around this by putting multiple people in the first argument: \verb|\partners{One, Two}{Three}|. There is currently no support for zero or one partners; to do that, you will need to edit the \verb|.cls| file yourself.

\section{Scientific Stuff}
\subsection{Figures}
Figure support does not go much beyond some basic formatting on top of the standard \verb|graphicx| package, which is included by default. For example, you can include the provided \verb|sample-graph.pdf| like this:
\begin{verbatim}
  \fig{sample-graph.pdf}
  {The $textrm{sinc}(x, y)$ function plotted with \texttt{gnuplot}.}{sinc-plot}
\end{verbatim}

In general, this command takes the form:
\begin{verbatim}
  \fig{placement}{file}{caption}{label}
\end{verbatim}

Where \verb|placement| is an optional argument corresponding to a \LaTeX float placement option; it defaults to \verb|[h!]|. The above example produces the shown figure. Actually, I think it's best to not use the \verb|fig| command and just insert a figure like a normal person.

\fig{sample-graph.pdf}
{The $\textrm{sinc}(x, y)$ function plotted with \texttt{gnuplot}.}{sinc-plot}

The figure can then be referenced in the document as Fig.~\ref{sinc-plot} by using the standard \verb|\ref{sinc-plot}| (you may need to recompile). The creation of such figures is beyond the scope of this document, but some free tools worth consideration are \verb|matplotlib| and \verb|gnuplot|.

\subsection{Tables}
Table formatting is provided by the custom \verb|chemtable| environment (which is really just a wrapper for a \verb|table| + \verb|threeparttable| + some formatting). This environment takes two arguments: the table title and the table label. You can define a new table like this:
\begin{verbatim}
  \begin{chemtable}{Table Title}{table-label}
    \begin{tabular}{c c c}
      \hline
      Year & \% Projected Growth & Net Profit (\$)\\
      \hline
      2010 & 5.3 & 135,789.54 \\
      2011 & 4.6 & 158,134.87 \\
      2012 & 8.9 & 179,020.40 \\
      \hline
    \end{tabular}
    \chemnotes{Values taken from absolutely nowhere.}  % creates a table note
  \end{chemtable}
\end{verbatim}

This results in the formatted Table~\ref{table-label}.
\begin{chemtable}{Table Title}{table-label}
  \begin{tabular}{c c c}
    \hline
    Year & \% Projected Growth & Net Profit (\$)\\
    \hline
    2010 & 5.3 & 135,789.54 \\
    2011 & 4.6 & 158,134.87 \\
    2012 & 8.9 & 179,020.40 \\
    \hline
  \end{tabular}
  \chemnotes{Values taken from absolutely nowhere.}  % creates a table note
\end{chemtable}

You can then refer to this table in the text as Table~\ref{table-label} by using \verb|\ref{table-label}|. Note that you may have to compile your document twice for this to work.
Of course, if you need more customization, you can easily define your own tables within the standard \verb|table| environment. If you simply need to specify the placement of the table, you can do that with the optional argument \verb|placement|, like this:
\begin{verbatim}
  \begin{chemtable}{h!}{Table Title}{table-label}
    <table content>
  \end{chemtable}
\end{verbatim}

\subsection{SI Units}
SI unit formatting is provided by the \verb|siunitx| package, which is already included by default. You can use this package to make units look consistent within equations:
\begin{equation}
  {\Delta}{m}=m_{f}-m_{i}=7.69~\si{\micro\gram}-4.23~\si{\micro\gram}=3.46~\si{\micro\gram}
\end{equation}

Which can be obtained with the following:
\begin{verbatim}
  \begin{equation}
    {\Delta}{m}=m_{f}-m_{i}
    =7.69~\si{\micro\gram}-4.23~\si{\micro\gram}=3.46~\si{\micro\gram}
  \end{equation}
\end{verbatim}

This is a little more convenient than simply wrapping all units within a \verb|textrm| and provides logical support for characters like $\mu$.

\subsection{Chemical Formulae}
\verb|lab_report| also includes the \verb|chemfig| and \verb|mhchem| packages to allow you make chemical formulae and equations with ease. Generally, a quick inline formula can be made with \verb|mhchem| like this: \verb|\ce{H_2O}| gives \ce{H_2O}. A more complicated formula can be typeset using the \verb|chemfig| package.

\section{Standard References}
Since this document is primarily for chemistry labs, it comes with the standard EPA Substance Registry Service and the Chemical Safety Data Sheet Search citations built in. You can create a standard bibliography section with these two entries and dates automatically assigned to the day of the lab. Extending this section involves using the \verb|chemrefs| environment and the \verb|chemcite{label}{name}{url}| macro, which look like this:

\begin{verbatim}
  \begin{chemrefs}
    \chemcite{wiki}{Wikipedia}{https://www.wikipedia.org/}
    % of course, you shouldn't cite Wikipedia in a real lab...
  \end{chemrefs}
\end{verbatim}

This will automatically create a references section with EPA SRS and Safety Data Sheet Search already cited, in addition to however many citations you make with the \verb|chemcite| command. Note that all special characters are already escaped with this command: you do not need to manually type \verb|\_| to create \_, for example. The result of this is the ``References'' section at the end of this document.

\begin{chemrefs}
  \chemcite{wiki}{Wikipedia}{https://www.wikipedia.org/}
\end{chemrefs}
\end{document}
